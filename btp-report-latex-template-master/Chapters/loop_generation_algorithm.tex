\chapter{Loop Generation Algorithm}
We have given an algorithm to find the sets of points that can determine the optimal partitions. Now we give an algorithm to generate actual loops given these sets of points. \\

\section{Algorithm}


generate\_loop (A, B, boundaries, l) \{ \\

\noindent /* \\
generate\_loop takes two sets of points as input and outputs the loops representing P(A, B) \\

\noindent A, B are the sets of points representing partition P(A, B) \\
boundaries contains the upper limit of the index variables \\
l denotes the current loop nest level. \\
*/ \\

\indent	A = sort\_and\_dominate(A) \\
\indent	B = sort\_and\_dominate(B) \\

\indent	//BASE CASE \\
\indent	if A and B have points with only 1 dimension then \{ \\
\indent \indent		start = A.first()[0]; \\
\indent		if (B is empty) \{ \\
\indent\indent			end = boundarie[l]; \\
\indent		\} \\
\indent		else \{ \\
\indent\indent			end = B.first()[0] - 1; \\
\indent		\} \\
\indent		put forall $i_l$ = start to end \\
\indent		put all the statements to be executed \\
\indent		put endall \\

\indent		return; \\
\indent	\} \\
	
\indent	A' = empty \\
\indent	B' = empty \\
\indent	start = min (A.first()[0], B.first()) \\
	
\indent	while (A is not empty or B is not empty) \{ \\
		
\indent\indent		while (A.first()[0] == start) \{ \\
\indent\indent\indent			remove first point from A and put it in A' after removing the first dimension from it \\
\indent\indent\indent			// eg: if first point is (2, 4, 1), then put (4, 1) in A' \\
\indent\indent		\} \\
\indent\indent		while (B.first()[0] == start) \{ \\
\indent\indent\indent			remove first point from B and put it in B' after removing the first dimension from it \\
\indent\indent		\} \\

\indent\indent		if (both A and B are empty) \{ \\
\indent\indent\indent			end = boundaries[l]; \\
\indent\indent		\} \\
\indent\indent		else \{ \\
\indent\indent\indent			end = min(A.first()[0], B.first()[0]) - 1; \\
\indent\indent		\} \\

\indent\indent		put forall $i_l$ = start to end \\
\indent\indent		generate\_loop (A', B', boundaries, l+1); \\
\indent\indent		put endall \\

\indent\indent		start = end + 1 \\
\indent	\} \\
\} \\	

sort\_and\_dominate() sorts all the points on the first dimension and also removes redundant points, i.e., if a,b belongs A and a <= b then b is removed from A.

