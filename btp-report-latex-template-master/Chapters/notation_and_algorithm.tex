\chapter{Optimal Partitioning Algorithm}
Consider the case of constant dependence distances. Consider only those cases where all the dependence distances are positive. The algorithm can then be easily extended when all dependence distances are negative for a particular loop nest.
\section{Notation}
For a nested loop, the vector of loop index for each nest is called as an iteration point. For example, consider a doubly nested loop with outer loop index i and inner loop index j. Then iteration point a = (2, 3) means that at this point, i = 2 and j = 3. Also $a_1 = 2$, $a_2 = 3.$ \\

Iteration point a is said to dominate iteration point b if\\
 $ \forall k,  a_k \leq b_k$ \\
We denote it by $a \preceq b$.\\
This creates a partial order on the set of iteration points.\\

We denote a partition by two sets of points, set A and set B, belonging to the iteration space. This partition is represented by P(A,B). \\

P(A,B) is the set of all iteration points that are dominated by at least one iteration point in set A and are not dominated by any iteration point from set B \\
$P(A,B) = \{x | ( \exists a \in A, a \preceq x) and (\forall b \in B, not(b \preceq x))\}$ \\

Thus if we have sets $A_0, A_1, ... A_k$ then we have k partitions: \\
$P(A_0,A_1), P(A_1, A_2), P(A_2, A_3), ... , P(A_{k-1}, A_k)$. \\

\subsection{Example}

For the example in figure~\ref{fig:partition_eg}, the sets are as given below: \\
$A_0 = \{(0,0)\}$ \\
$A_1 = \{(1,4), (2,2), (3,1)\}$ \\
$A_2 = \{(2,8), (3,6), (4,4), (5,3), (6,2)\}$ \\
$A_3 = \{(5,8), (6,6), (7,5), (8,4), (9,3)\}$ \\
$A_4 = \{(8,8), (9,7)\}$ \\
$A_5 = \{\}$ \\

\begin{figure}
\caption{Partitions by our algorithm}
\label{fig:partition_eg}
\centering \includegraphics[width=0.5\textwidth]{Figures/fig2.jpg}
\end{figure}

%\ref{append:labelforappendix}

\section{Algorithm to generate the sets of points}

We give an iterative algorithm in which, given a set of points, it generates the next set.
Let the sets be called $A_0, A_1, .. ,A_k$ \\

Consider the set of all the dependence distances of the loop. 
Create a minimal set of dependence distances by removing any dependence distance that is dominated by another dependence distance in the set.
Thus in the minimal set, no dependence distance dominates any other dependence distance.
Call this set D. \\

\noindent
$A_0 = \{(0,0,..,0)\}$ /* Vector containing as many 0s as the total loop nest. First element corresponds to outermost loop and so on. */ \\
$iter = 0$ \\
while($A_{iter}$ is not empty) \{ \\
\indent	iter++ \\
\indent	$B = \{\}$ \\
\indent	$ \forall a \in A_{iter-1}, \forall d \in D$, insert a+d in B if a+d lies inside the loop bounds\\
\indent	Create a minimal set of B, by removing an iteration point from B if there exists another iteration point dominating it.\\
\indent	Thus in the final set obtained no iteration point dominates another point.\\
\indent	Set $A_{iter}$ as this minimal set obtained\\
\} \\

\subsection{Example}

\noindent For example consider the following code: \\
DO I = 0 to 9 \\
\indent DO J = 0 to 9 \\
\indent \indent A[I+1, J+4] = B[I, J] \\
\indent \indent C[I+2, J+2] = A[I, J] \\
\indent \indent B[I+3, J+1] = C[I, J] \\
\indent ENDO \\
ENDO \\

\noindent In this case, \\
$D = \{(1,4), (2,2), (3,1)\}$ \\
The above algorithm apllied to this example is shown in table~\ref{table:iterations}. \\
The partitions generated are depicted in figure~\ref{fig:partition_eg}. \\

\begin{table}
\caption {Iterations of the algorithm applied to example}
\label{table:iterations}
\begin{tabular}{|c| L{5.75cm} | L{5.75cm} | }

\hline

\bf iter &
\bf B &
\bf $A_{iter}$ \\ \hline

\bf 0 &
- &
$\{(0,0)\}$ \\ \hline

\bf 1 &
$\{(1,4), (2,2), (3,1)\}$ &
$\{((1,4), (2,2), (3,1)\}$ \\ \hline

\bf 2 &
$\{(2,8), (3,6), (4,5), (3,6), (4,4),$ $(5,3), (4,5), (5,3), (6,2)\}$ &
$\{(2,8), (3,6), (4,4), (5,3), (6,2)\}$ \\ \hline

\bf 3 &
$\{(5,9), (5,8), (6,7), (5,8), (6,6),$ $(7,5), (6,7), (7,5), (8,4), (7,6),$ $(8,4), (9,3)\}$ &
$\{(5,8), (6,6), (7,5), (8,4), (9,3)\}$ \\ \hline

\bf 4 &
$\{(8,9), (8,8), (9,7), (8,9), (9,7),$ $(9,8)\}$ &
$\{(8,8), (9,7)\}$ \\ \hline

\bf 5 &
$\{\}$ &
$\{\}$ \\ \hline

\end{tabular}
\end{table}

\section{Proof of Correctness}
We need to prove that a source-sink pair can never lie in the same partition. We prove this by contradiction. \\

Suppose there exists a source u and sink v that lie in the same partition P(A,B).
Let the dependence distance between them be d.\\
Thus $v = u + d$\\
Consider $d' \preceq d$ and $d' \in D$\\

Lemma 1: There always exists such d'\\

\noindent Let $ v' = u + d'$\\
Thus, u, v' is another source-sink pair\\

Lemma 2: If $u, v \in P(A,B)$  then $u , v' \in P(A,B)$.\\

\noindent As, $u, v' \in P(A,B), \exists a \in A, a \preceq u$ \\
By the method in which partitions are created, \\
$\exists b \in B, b \preceq a + d'$ \\
$a \preceq u$ \\
$ \Rightarrow a + d' \preceq u + d'$ \\
$\Rightarrow b \preceq a + d' \preceq v'$ \\
$\Rightarrow b \preceq v'$ \\
Thus it implies v does not belong to P(A,B) which is a contradiction. \\

Thus, our assumption is false. Therefore there does not exist any source-sink pair belonging to same partition.\\

\subsection{Proof of Lemma 1}
If $d \in D$ then $d = d'$\\
else as d was removed from the set of dependence distances, there must exist another dependence distance in D dominating d. Let that be d' \\


\subsection{Proof of Lemma 2}
$u,v \in P(A,B)$ \\
$\Rightarrow \exists a \in A , a \preceq u$ \\
$ \Rightarrow a \preceq u \preceq v'$ \\
$ \Rightarrow a \preceq v' $ \\

Suppose $v' \notin P(A,B)$. \\
$ \Rightarrow \exists b \in B, b \preceq v'$ \\
$ d' \preceq d $ \\
$ \Rightarrow d' + u \preceq d + u $ \\
$ \Rightarrow v' \preceq v$ \\
$ (b \preceq v'$ and $v' \preceq v) \Rightarrow b \preceq v$ \\ 
$ \Rightarrow v \notin P(A, B)$ \\
Therefore our assumption is false. \\
$v' \in P(A,B)$ \\

\section{Proof of Optimality}
We need to prove that even if a single iteration point from a partition is added to its previous partition, it will lead to at least one source-sink pair belonging to same partition. \\

Suppose the partitions we created are not optimal. 
Therefore there exists a point y which is in the next partition of P(A,B) which can be added to P(A,B). i.e y has no source in P(A,B). \\

\noindent $y \notin P(A,B) \Rightarrow  \exists b \in B, b \preceq y$ \\
By the method in which partitions are created, \\
$ \exists d \in D$ and $a \in A, a + d = b$ \\
Let $x = y - d$ \\
$ b \preceq y \Rightarrow b - d \preceq y - d$ \\
$ \Rightarrow a \preceq x$ \\
This means that $x \in P(A,B)$ \\
x and y form a source-sink pair. Therefore there is a source of y in P(A,B).
Therefore our assumption is false. \\

Thus partitions created are optimal.\\